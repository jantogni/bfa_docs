\documentclass[10pt]{article}
\usepackage[utf8]{inputenc}
\usepackage[english, activeacute]{babel}
\usepackage{geometry}
\usepackage{url}
\usepackage{algorithm}
\usepackage{amsmath}
\usepackage[noend]{algpseudocode}
\usepackage{setspace}
%\geometry{tmargin=3.0cm, lmargin=3.0cm, rmargin=2.5cm, bmargin=3.0cm}

\usepackage{listings}
\usepackage{color}
\definecolor{gray}{rgb}{0.4,0.4,0.4}
\definecolor{darkblue}{rgb}{0.0,0.0,0.6}
\definecolor{cyan}{rgb}{0.0,0.6,0.6}

\lstset{
  basicstyle=\ttfamily,
  columns=fullflexible,
  showstringspaces=false,
  commentstyle=\color{gray}\upshape
}

\lstdefinelanguage{XML}
{
  morestring=[b]",
  morestring=[s]{>}{<},
  morecomment=[s]{<?}{?>},
  stringstyle=\color{black},
  identifierstyle=\color{darkblue},
  keywordstyle=\color{cyan},
  morekeywords={xmlns,version,type}% list your attributes here
}

\newcommand\pythonstyle{\lstset{
language=Python,
basicstyle=\ttm,
otherkeywords={self},             % Add keywords here
keywordstyle=\ttb\color{deepblue},
emph={MyClass,__init__},          % Custom highlighting
emphstyle=\ttb\color{deepred},    % Custom highlighting style
stringstyle=\color{deepgreen},
frame=tb,                         % Any extra options here
showstringspaces=false            % 
}}

\makeatletter
\def\BState{\State\hskip-\ALG@thistlm}
\makeatother

\title{
\center{\emph{Flux Estimation Algorithm} \\}
\author{
        Ruediger Kneissl, Jonathan Antognini\\
%\date{Santiago, \today}
}}

\begin{document}
\maketitle

\section{General Specifications}

The Flux Estimation Algorithm was provided by Ruediger. The objective is to estimate
a flux of certain source (name) given a date and frequency parameters
\footnote{\url{http://twiki.csrg.cl/twiki/bin/view/LIRAE/SourceCatalogueVO}}.

Webservice details:
\begin{itemize}
 \item Rest Webservice in Java 1.6 (at the moment) using Spring MVC (org.springFramework)
 \item Database queries: XmlRpcClient (org.apache), searchMeasurements103
 \item Output: VOTable, Java Streaming Writer (voi.vowrite)
\end{itemize}

\section{Algorithm Cases}

The estimation of the flux was divided in 3 cases: 10 days time windows, 4
months time windows and the average in time. The following subsections describe
each procedure:

\subsection{Ten days Time Windows}
\begin{algorithm}
\caption{bestFluxAlgorithm}\label{10days}
\begin{algorithmic}[1]
\Procedure{10daysTimeWindows}{time, frequency}
\State $\textit{sameBand} \gets \text{Vector with elements in the same band}$
\If {$sameBand.size() > \textit{1}$}: 
\State $closestInTime \gets \text{Closest measurement in sameBand Vector}$.
\Statex
\State \Return {$\text{closestInTime.flux * $\left(\frac{closestInTime.frequency}{frequency}\right)^{alpha}$}$}
\Else
\State \Return {$\textit{Null value}$}
\EndIf
\EndProcedure
\end{algorithmic}
\end{algorithm}

\subsection{Average in time}
\begin{algorithm}
\caption{bestFluxAlgorithm}\label{4months}
\begin{algorithmic}[1]
\Procedure{Average in time}{time, frequency}
\State $\textit{AllMes} \gets \text{Vector with all measurement}$
\Statex
\State $\textit{averageFrequency} \gets \frac{\Sigma AllMes.frequency(i)}{AllMes.size()}$
\Statex
\State $\textit{averageFlux} \gets \frac{\Sigma AllMes.flux(i)}{AllMes.size()}$
\Statex
\State $\textit{standDevFlux} \gets \frac{\Sigma (AllMes.flux(i) - averageFlux)^2}{AllMes.size()}$
\Statex
\State \Return {$\textit{averageFrequency, averageFlux, standDevFlux}$}
\EndProcedure
\end{algorithmic}
\end{algorithm}

\section{Four months Time Windows}
In the 4 months time windows we try to estimate the flux using:
$$ \mathbf{Flux} = (a(t-t_0)+b)*\left(\frac{f_0}{f}\right)^\alpha $$
where: $a,b,\alpha$ are parameters of the non-linear regression; $t_0, f_0$ are
the time and frequency of the source.

In order to aproximate the parameters of the model, we proposed a
Levenber-Marquadt optimization. 

\subsection{Levenberg-Marquardt (LM)}
\textbf{Problem} 

The primary application of the Levenberg–Marquardt algorithm is in the least
squares curve fitting problem: given a set of m empirical datum pairs of
independent and dependent variables, ($x_i, y_i$), optimize the parameters
$\beta$ of the model curve $f(x,\beta)$ so that the sum of the squares of the
deviations

$$ f(x_i, \beta + \delta) \approx \sum\limits_{i=1}^m [y_i - f(x_i, \beta)]^2 $$

\noindent\textbf{Solution} 

$$ (\mathbf{J}^{T}\mathbf{J} + \lambda\mathbf{I})\delta = \mathbf{J}^{T}[y - f(\beta)] $$
where: $\mathbf{J}$ is the gradient matrix, $\mathbf{I}$ is the identity matrix, $\delta$ adjust of vector $\beta$.


\noindent\textbf{Gradient: $\mathbf{J}$} 

$$ \mathbf{J} = \nabla f = \frac{\partial f}{\partial \beta} $$ 

where: $\beta = [a, b, \alpha]$. In this case, each partial derivate:
\begin{align*}
    \frac{\partial f}{\partial a}           &= (t - t_0)\left(\frac{f_0}{f}\right)^\alpha \\
    \frac{\partial f}{\partial b}           &= \left(\frac{f_0}{f}\right)^\alpha \\
    \frac{\partial f}{\partial \alpha}      &= (a(t-t_0)+b)*\left(\frac{f_0}{f}\right)^\alpha ln\left(\frac{f_0}{f}\right)
\end{align*}

\subsection{Weighted Levenberg-Marquardt}
A variation of the initial problem of fitting is proposed:

$$ f(x_i, \beta + \delta) \approx \sum\limits_{i=1}^m \left[\frac{y_i - f(x_i, \beta)}{w_i} \right]^2 $$

\noindent where $w_i$ is the flux uncertainty of the measure $i$.

\noindent The new solution for this problem is:

$$ (\mathbf{J}^{\text{T}}\mathbf{W}\mathbf{J} + \lambda~\text{diag}(\mathbf{J}^{\text{T}}\mathbf{W}\mathbf{J})) \delta = \mathbf{J}^{\text{T}}\mathbf{W}[y - f(\beta)] $$

\subsection{Error Analysis}

The standard parameter errors are given by:

$$ \sigma_p = \sqrt{\text{diag}([\mathbf{J}^{\text{T}}\mathbf{W}\mathbf{J}]^{-1})} $$

\noindent where $\sigma_p = [\sigma_a, \sigma_b, \sigma_{\alpha}] $


\noindent The standard errors for the fit are given by:

$$ \sigma_{\hat{y}} = \sqrt{\text{diag}(\mathbf{J}[\mathbf{J}^{\text{T}}\mathbf{W}\mathbf{J}]^{-1}\mathbf{J}^{\text{T}})} $$


\begin{algorithm}
\caption{bestFluxAlgorithm}\label{4months}
\begin{algorithmic}[1]
\Procedure{4 months time windows}{time, frequency}
\State $\textit{iterations} \gets \text{Number of iterations of LM}$
\Statex
\State $\textit{t} \gets \text{Time adjusted within [0, 10]}$
\Statex
\State $\textit{$\beta_0$} \gets \text{Initial values [5, 1, -0.7]}$
\Statex
\State $\textit{$\lambda$} \gets \text{Small value: 0.01}$
\Statex
\For{\text{$i=0 ; i<iterations$}}
\Statex
\State $\textit{$\delta$} \gets \text{Solution of $ (\mathbf{J}_i^{T}\mathbf{J}_i + \lambda\mathbf{I})\delta = \mathbf{J}_i^{T}[y - f(\beta_i)] $}$
\Statex
\State $\textit{$\beta_{i+1}$} \gets \text{$ \beta_i + \delta $}$
\Statex
\EndFor
\State \Return {$\textit{$\beta[1] \to b$}$}
\EndProcedure
\end{algorithmic}
\end{algorithm}

\section{Web service}
\subsection{Usage}
The algorithm is working as web service and support HTTP GET request. At the moment we have the
prototype of this implementation in:

\begingroup
\fontsize{8pt}{10pt}\selectfont
\begin{verbatim}
http://vo-prototype-test.sco.alma.cl:8080/bfs-0.2/ssap?
\end{verbatim}
\endgroup

\noindent An example of usage through web browser
\begingroup
\fontsize{8pt}{10pt}\selectfont
\begin{verbatim}
http://vo-prototype-test.sco.alma.cl:8080/bfs-0.2/ssap?NAME=3c279&DATE=04-Apr-2014&FREQUENCY=231.435E9&TEST=false&VERBOSE=true
\end{verbatim}
\endgroup

\noindent An example of usage through command line (curl):
\begingroup
\fontsize{8pt}{10pt}\selectfont
\begin{verbatim}
curl --request GET 'http://vo-prototype-test.sco.alma.cl:8080/bfs-0.2/ssap?NAME=3c279&DATE=04-Apr-2014&FREQUENCY=231.435E9&TEST=false&VERBOSE=true'
\end{verbatim}
\endgroup

\noindent The parameters for the webservice are"
\begin{itemize}
 \item NAME: source name in the source catalogue database. It not need to be quoted, also it can contain special characters: \emph{+-\_}
 \item FREQUENCY: frequency used to estimate the flux. Format in double, for example: \emph{231.435E9}
 \item DATE: date used to estimate the flux. Format in string, for example: \emph{dd-mmm-yyyy, 04-Apr-2014}
 \item TEST: atribute defined \emph{false} by default, it not necessary to be in the query. If this parameter is defined true, the algorithm will not take measurements in the same day.
 \item VERBOSE: atribute defined \emph{false} by default, it not necessary to be in the query. If this parameter is defined truel, the output will contain a field with all the loggin produced by the algorithm.
\end{itemize}


\subsection{Output}
\noindent Both responses are VOTable (xml file). An XML output example:
\lstset{language=XML}
\begin{lstlisting}
<?xml version="1.0"?>
<!DOCTYPE VOTABLE SYSTEM "http://us-vo.org/xml/VOTable.dtd">
<VOTABLE  >
   <DESCRIPTION>
   Flux estimation with 4 months window
   </DESCRIPTION>
   <RESOURCE  >
      <TABLE >
         <FIELD datatype="char"  name="SourceName"  arraysize="16"  />
         <FIELD datatype="double"  width="10"  name="Frequency"  />
         <FIELD datatype="char"  name="Date"  arraysize="32"  />
         <FIELD datatype="double"  width="10"  name="FluxDensity"  />
         <FIELD datatype="double"  width="10"  name="FluxDensityError"  />
         <FIELD datatype="double"  width="10"  name="SpectralIndex"  />
         <FIELD datatype="double"  width="10"  name="SpectralIndexError"  />
         <FIELD datatype="double"  width="10"  name="error2"  />
         <FIELD datatype="double"  width="10"  name="error3"  />
         <FIELD datatype="double"  width="10"  name="error4"  />
         <FIELD datatype="int"  width="10"  name="warning"  />
         <FIELD datatype="int"  width="10"  name="notms"  />
         <FIELD datatype="char"  name="verbose"  arraysize="256000"  />
         <DATA>
            <TABLEDATA>
            <TR>
               <TD>3c279</TD>
               <TD>2.31435E11</TD>
               <TD>Fri Apr 04 00:00:00 CLST 2014</TD>
               <TD>9.586108872789792</TD>
               <TD>0.21581694351063035</TD>
               <TD>-0.5762192833875143</TD>
               <TD>0.02988085761954256</TD>
               <TD>0.060135781652679104</TD>
               <TD>0.5638507073500892</TD>
               <TD>0.0884885466683922</TD>
               <TD>141</TD>
               <TD>1</TD>
               <TD>empty</TD>
            </TR>
            </TABLEDATA>
         </DATA>
      </TABLE>
   </RESOURCE>
</VOTABLE>
\end{lstlisting}

\noindent where:
\begin{itemize}
    \item SourceName: Source Name
    \item Frequency: Frequency used in the estimation
    \item Date: Date used in the estimation
    \item FluxDenisty: Flux Density estimated (four months time window)
    \item FluxDenistyError: Flux Denisty Error calculated from the fit (B in the model)
    \item SpectralIndex: Estimated Spectral Index 
    \item SpectralIndexError: Spectral Index error calculated from the fit ($\alpha$ in the model)
    \item error2: $\sigma_{\bar{y}}$ %$\frac{\sqrt{\frac{\sum\sigma_i^2}{N}}}{\sqrt{N}}$
    \item error3: $\sigma_{\hat{y}}$
    \item error4: error from monte carlo simulation.
    \item warning: string with 3 character: 
        \begin{itemize} 
            \item str[0] correspond to amount of measurement availables, 1: for 1 measurement, 2: for 2 measurement, 3: for 3 or more measurements.
            \item str[1] it says if there are 1 measurement within 10 days time window in the same band.
            \item str[2] it says if there are at least 1 measurement before and after (1 true, 0 false),
            \item By default, all 3 character are 4.
        \end{itemize}
    \item notms: 0 not enough measurement ($<$3), 1 otherwise.
    \item verbose: string field with the loggin of the algorithm. By default \emph{empty}.
\end{itemize}

\section{Client}
In order to simplify the intercation with the webservice, a client was created.
This client has two dependencies: astropy and numpy. An example of usage:

\begingroup
\fontsize{8pt}{10pt}\selectfont
\lstset{language=Python}
\begin{lstlisting}
def main():
    #Using default URL
    flux = fluxEstimation('3c279', '04-Apr-2014', '231.435E9,240.435E9', verbose = 'false', test = 'false')

    #Using custom URL
    #bender_url = 'http://bender.csrg.cl:2121/bfs-0.1/ssap'
    #flux = fluxEstimation('3c279', '04-Apr-2014', '231.435E9,240.435E9', url = DEF_URL_localhost,240.435E9)

    flux.performQuery()
    flux.parseResponse()

    #Iterate over all rows
    for row in flux.data:
        print "----------------------------------"
        print row.asDict()
        #print row.asList()
    print "----------------------------------"

if __name__ == "__main__":
    main()
\end{lstlisting}
\endgroup

\end{document}
