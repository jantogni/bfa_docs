\documentclass[10pt]{article}
\usepackage[utf8]{inputenc}
\usepackage[english, activeacute]{babel}
\usepackage{geometry}
\usepackage{url}
\usepackage{algorithm}
\usepackage{amsmath}
\usepackage[noend]{algpseudocode}
\usepackage{setspace}
%\geometry{tmargin=3.0cm, lmargin=3.0cm, rmargin=2.5cm, bmargin=3.0cm}

\usepackage{listings}
\usepackage{color}
\definecolor{gray}{rgb}{0.4,0.4,0.4}
\definecolor{darkblue}{rgb}{0.0,0.0,0.6}
\definecolor{cyan}{rgb}{0.0,0.6,0.6}

\usepackage{hyperref}
\hypersetup{
    colorlinks,
    citecolor=black,
    filecolor=black,
    linkcolor=black,
    urlcolor=black
}

\lstset{
  basicstyle=\ttfamily,
  columns=fullflexible,
  showstringspaces=false,
  commentstyle=\color{gray}\upshape
}

\lstdefinelanguage{XML}
{
  morestring=[b]",
  morestring=[s]{>}{<},
  morecomment=[s]{<?}{?>},
  stringstyle=\color{black},
  identifierstyle=\color{darkblue},
  keywordstyle=\color{cyan},
  morekeywords={xmlns,version,type}% list your attributes here
}

\newcommand\pythonstyle{\lstset{
language=Python,
basicstyle=\ttm,
otherkeywords={self},             % Add keywords here
keywordstyle=\ttb\color{deepblue},
emph={MyClass,__init__},          % Custom highlighting
emphstyle=\ttb\color{deepred},    % Custom highlighting style
stringstyle=\color{deepgreen},
frame=tb,                         % Any extra options here
showstringspaces=false            % 
}}

\makeatletter
\def\BState{\State\hskip-\ALG@thistlm}
\makeatother

\title{
\center{\emph{Flux Estimation Algorithm} \\}
\author{
        Ruediger Kneissl, Jonathan Antognini\\
%\date{Santiago, \today}
}}

\begin{document}
\maketitle

\tableofcontents

\section{General Specifications}

The Flux Estimation Algorithm was provided by Ruediger. The objective is to estimate
a flux of certain source (name) given a date and frequency parameters
\footnote{\url{http://twiki.csrg.cl/twiki/bin/view/LIRAE/SourceCatalogueVO}}.

Web service details:
\begin{itemize}
 \item Rest Web service in Java JDK 1.7 using Spring MVC (org.springFramework)
 \item Database queries: XmlRpcClient (org.apache), searchMeasurements103
 \item Output: VOTable, Java Streaming Writer (voi.vowrite)
\end{itemize}

\section{Algorithm Cases}

The estimation of the flux was divided in 3 cases: 10 days time windows, 4
months time windows and the average in time. The following subsections describe
each procedure:

\subsection{Ten days Time Windows}
\begin{algorithm}
\caption{bestFluxAlgorithm}\label{10days}
\begin{algorithmic}[1]
\Procedure{10daysTimeWindows}{time, frequency}
\State $\textit{sameBand} \gets \text{Vector with elements in the same band}$
\If {$sameBand.size() > \textit{1}$}: 
\State $closestInTime \gets \text{Closest measurement in sameBand Vector}$.
\Statex
\State \Return {$\text{closestInTime.flux * $\left(\frac{frequency}{closestInTime.frequency}\right)^{alpha}$}$}
\Else
\State \Return {$\textit{Null value}$}
\EndIf
\EndProcedure
\end{algorithmic}
\end{algorithm}

\subsection{Average in time}
\begin{algorithm}[H]
\caption{bestFluxAlgorithm}\label{4months}
\begin{algorithmic}[1]
\Procedure{Average in time}{time, frequency}
\State $\textit{AllMes} \gets \text{Vector with all measurement}$
\Statex
\State $\textit{averageFrequency} \gets \frac{\Sigma AllMes.frequency(i)}{AllMes.size()}$
\Statex
\State $\textit{averageFlux} \gets \frac{\Sigma AllMes.flux(i)}{AllMes.size()}$
\Statex
\State $\textit{standDevFlux} \gets \frac{\Sigma (AllMes.flux(i) - averageFlux)^2}{AllMes.size()}$
\Statex
\State \Return {$\textit{averageFrequency, averageFlux, standDevFlux}$}
\EndProcedure
\end{algorithmic}
\end{algorithm}

\subsection{Four months Time Windows}
In the 4 months time windows we try to estimate the flux using:
$$ \mathbf{Flux} = (a(t-t_0)+b)*\left(\frac{f}{f_0}\right)^\alpha $$
where: $a,b,\alpha$ are parameters of the non-linear regression; $t_0, f_0$ are
the time and frequency of the source.

In order to approximate the parameters of the model, we proposed a
Levenberg-Marquadt optimization. 

\subsubsection{Levenberg-Marquardt (LM)}
\textbf{Problem} 

The primary application of the Levenberg–Marquardt algorithm is in the least
squares curve fitting problem: given a set of m empirical datum pairs of
independent and dependent variables, ($x_i, y_i$), optimize the parameters
$\beta$ of the model curve $f(x,\beta)$ so that the sum of the squares of the
deviations

$$ f(x_i, \beta + \delta) \approx \sum\limits_{i=1}^m [y_i - f(x_i, \beta)]^2 $$

\noindent\textbf{Solution} 

$$ (\mathbf{J}^{T}\mathbf{J} + \lambda\mathbf{I})\delta = \mathbf{J}^{T}[y - f(\beta)] $$
where: $\mathbf{J}$ is the gradient matrix, $\mathbf{I}$ is the identity matrix, $\delta$ adjust of vector $\beta$.


\noindent\textbf{Gradient: $\mathbf{J}$} 

$$ \mathbf{J} = \nabla f = \frac{\partial f}{\partial \beta} $$ 

where: $\beta = [a, b, \alpha]$. In this case, each partial derivate:
\begin{align*}
    \frac{\partial f}{\partial a}           &= (t - t_0)\left(\frac{f}{f_0}\right)^\alpha \\
    \frac{\partial f}{\partial b}           &= \left(\frac{f}{f_0}\right)^\alpha \\
    \frac{\partial f}{\partial \alpha}      &= (a(t-t_0)+b)*\left(\frac{f}{f_0}\right)^\alpha ln\left(\frac{f}{f_0}\right)
\end{align*}

\begin{algorithm}[H]
\caption{Non linear fit using levenberg-marquardt}\label{4months}
\begin{algorithmic}[1]
\Procedure{4 months time windows}{time, frequency}
\State $\textit{iterations} \gets \text{Number of iterations of LM}$
\Statex
\State $\textit{t} \gets \text{Time adjusted within [0, 10]}$
\Statex
\State $\textit{$\beta_0$} \gets \text{Initial values [5, 1, -0.7]}$
\Statex
\State $\textit{$\lambda$} \gets \text{Small value: 0.01}$
\Statex
\For{\text{$i=0 ; i<iterations$}}
\Statex
\State $\textit{$\delta$} \gets \text{Solution of $ (\mathbf{J}_i^{T}\mathbf{J}_i + \lambda\mathbf{I})\delta = \mathbf{J}_i^{T}[y - f(\beta_i)] $}$
\Statex
\State $\textit{$\beta_{i+1}$} \gets \text{$ \beta_i + \delta $}$
\Statex
\EndFor
\State \Return {$\textit{$\beta[1] \to b$}$}
\EndProcedure
\end{algorithmic}
\end{algorithm}

\subsubsection{Weighted Levenberg-Marquardt}
A variation of the initial problem of fitting is proposed:

$$ f(x_i, \beta + \delta) \approx \sum\limits_{i=1}^m \left[\frac{y_i - f(x_i, \beta)}{w_i} \right]^2 $$

\noindent where $w_i$ is the flux uncertainty of the measure $i$.

\noindent The new solution for this problem is:

$$ (\mathbf{J}^{\text{T}}\mathbf{W}\mathbf{J} + \lambda~\text{diag}(\mathbf{J}^{\text{T}}\mathbf{W}\mathbf{J})) \delta = \mathbf{J}^{\text{T}}\mathbf{W}[y - f(\beta)] $$

\begin{algorithm}[H]
\caption{Non linear fit using weighted levenberg-marquardt}\label{4months}
\begin{algorithmic}[1]
\Procedure{4 months time windows}{time, frequency}
\State $\textit{iterations} \gets \text{Number of iterations of LM}$
\Statex
\State $\textit{t} \gets \text{Time adjusted within [0, 10]}$
\Statex
\State $\textit{$\beta_0$} \gets \text{Initial values [5, 1, -0.7]}$
\Statex
\State $\textit{$\lambda$} \gets \text{Small value: 0.01}$
\Statex
\For{\text{$i=0 ; i<iterations$}}
\Statex
\State $\textit{$\delta$} \gets \text{Solution of $ (\mathbf{J}^{\text{T}}\mathbf{W}\mathbf{J} + \lambda~\text{diag}(\mathbf{J}^{\text{T}}\mathbf{W}\mathbf{J})) \delta = \mathbf{J}^{\text{T}}\mathbf{W}[y - f(\beta)] $}$
\Statex
\State $\textit{$\beta_{i+1}$} \gets \text{$ \beta_i + \delta $}$
\Statex
\EndFor
\State \Return {$\textit{$\beta[1] \to b$}$}
\EndProcedure
\end{algorithmic}
\end{algorithm}

\subsection{Error Analysis}
In order to give more information about the model fit, the algorithm calculates a list of error:

For each parameter, each error is given by the diagonal of the covariance matrix:

$$ \sigma_p = \sqrt{\text{diag}([\mathbf{J}^{\text{T}}\mathbf{W}\mathbf{J}]^{-1})} $$

\noindent where $\sigma_p = [\sigma_a, \sigma_b, \sigma_{\alpha}]$. The values present in the output of the algorithm are:
\begin{itemize}
    \item $\sigma_b$: FluxDensityError 
    \item $\sigma_{\alpha}$: SpetralIndexError
\end{itemize}

\noindent \textbf{Error2}: represent the difference between model and real data:

$$ \text{error2} = \frac{|f(\beta) - flux_i|}{N\sqrt{N}}  $$

\noindent \textbf{Error3}: standard error for the fit is given by:

$$ \sigma_{\hat{y}} = \text{error3} = \sqrt{\sum_{i=1}^{N}\frac{(\mathbf{J}[\mathbf{J}^{\text{T}}\mathbf{W}\mathbf{J}]^{-1}\mathbf{J}^{\text{T}})_{ii}}{N}} $$

\noindent \textbf{Error4}: Monte Carlo error. \\

\noindent \textbf{Model goodness}: it takes in consideration the differences
between the model and real data, as well as each individual error from the
data. The algorithm iterates over the number of measurement (3, 4, 5, 6 ... N),
and return the flux with the best model goodness:

$$ \sigma_i = \sqrt{ (Model - Data)^2 - \sigma_{data}^2 } $$
$$ \text{Model goodness} = \sum_{i=1}^{N}\frac{\sigma_i}{N\sqrt{N}} $$

\section{Web service}
\subsection{Usage}
The algorithm is working as web service and support HTTP GET request. Base url can change depending on the deployment environment (2015.8 in thi case)

\begingroup
\fontsize{8pt}{10pt}\selectfont
\begin{verbatim}
https://2015-08.asa-test.alma.cl/sc/flux?
\end{verbatim}
\endgroup

\noindent An example of usage through web browser
\begingroup
\fontsize{6pt}{10pt}\selectfont
\begin{verbatim}
https://2015-08.asa-test.alma.cl/sc/flux?NAME=3c279&DATE=04-Apr-2014&FREQUENCY=231.435E9
\end{verbatim}
\endgroup

\noindent An example of usage through command line (curl):
\begingroup
\fontsize{6pt}{10pt}\selectfont
\begin{verbatim}
curl --request GET 'https://2015-08.asa-test.alma.cl/sc/flux?NAME=3c279&DATE=04-Apr-2014&FREQUENCY=231.435E9'
\end{verbatim}
\endgroup

\noindent The parameters for the web service are"
\begin{itemize}
 \item NAME: source name in the source catalogue database. It not need to be quoted, also it can contain special characters: \emph{+-\_}
 \item FREQUENCY: frequency used to estimate the flux. Format in double, for example: \emph{231.435E9}
 \item DATE: date used to estimate the flux. Format in string, for example: \emph{dd-mmm-yyyy, 04-Apr-2014}
 \item TEST (OPTIONAL): by default \emph{false}. If this parameter is defined as true, the algorithm will not take measurements on the same day.
 \item VERBOSE (OPTIONAL): by default \emph{false}. If this parameter is defined as true, the output will contain a field with all the logging produced by the algorithm.
 \item MODEL (OPTIONAL): by default 0. Default calculate normal model goodness divided by $measurements^{1/2}$, if is defined as 1, calculate model goodness divided by $measurements^{1/2}$
 \item RESULT (OPTIONAL): by default 0. Default algorithm use a minimum of 8 measurement on 4 months time windows (which is the January algorithm), if it's defined as 1, will use 4 or more measurement on 4 months time windows (which is the March algorithm).
 \item WEIGHTED (OPTIONAL): By default \emph{false}, 4 month time windows will use \textbf{weighted} levenberg marquardt algorithm, if it's defined false, the algorithm will use non-wighted method.
\end{itemize}


\subsection{Output}
\noindent Both responses are VOTable (xml file). An XML output example:
\lstset{language=XML}
\begin{lstlisting}
<?xml version="1.0"?>
<!DOCTYPE VOTABLE SYSTEM "http://us-vo.org/xml/VOTable.dtd">
<VOTABLE  >
 <DESCRIPTION>
 Flux estimation with 4 months window
 </DESCRIPTION>
 <RESOURCE  >
  <TABLE >
   <FIELD datatype="char"  name="SourceName"  arraysize="16"  />
   <FIELD unit="Hz"  datatype="double"  width="10"  name="Frequency"  />
   <FIELD datatype="char"  name="Date"  arraysize="32"  />
   <FIELD unit="Jansky"  datatype="double"  width="10"  name="FluxDensity"  />
   <FIELD unit="Jansky"  datatype="double"  width="10"  name="FluxDensityError"  />
   <FIELD unit="Unitless"  datatype="double"  width="10"  name="SpectralIndex"  />
   <FIELD unit="Unitless"  datatype="double"  width="10"  name="SpectralIndexError"  />
   <FIELD unit="Jansky"  datatype="double"  width="10"  name="error2"  />
   <FIELD unit="Jansky"  datatype="double"  width="10"  name="error3"  />
   <FIELD unit="Jansky"  datatype="double"  width="10"  name="error4"  />
   <FIELD datatype="int"  width="10"  name="warning"  />
   <FIELD datatype="int"  width="10"  name="notms"  />
   <FIELD datatype="char"  name="verbose"  arraysize="256000"  />
   <DATA>
    <TABLEDATA>
    <TR>
     <TD>3c279</TD>
     <TD>2.31435E11</TD>
     <TD>Fri Apr 04 00:00:00 EDT 2014</TD>
     <TD>9.612161844649648</TD>
     <TD>0.0</TD>
     <TD>-0.5974024101823433</TD>
     <TD>0.0</TD>
     <TD>0.0</TD>
     <TD>0.0</TD>
     <TD>100.0</TD>
     <TD>311</TD>
     <TD>-1</TD>
     <TD>empty</TD>
    </TR>
    </TABLEDATA>
   </DATA>
  </TABLE>
 </RESOURCE>
</VOTABLE>
\end{lstlisting}

\noindent where:
\begin{itemize}
    \item SourceName: Source Name
    \item Frequency: Frequency used in the estimation
    \item Date: Date used in the estimation
    \item FluxDenisty: Flux Density estimated (b in the model)
    \item FluxDenistyError: Flux Density Error calculated from the fit ($\sigma_b$ in the model)
    \item SpectralIndex: Estimated Spectral Index ($\alpha$ in the model) 
    \item SpectralIndexError: Spectral Index error calculated from the fit ($\sigma_{\alpha}$ in the model)
    \item error2: $\sigma_{\bar{y}}$: $\frac{\sqrt{\sum\sigma_i^2}}{N}$
    \item error3: $\sigma_{\hat{y}}$: $\sqrt{\text{Average\:}(\text{diag}(\mathbf{J}[\mathbf{J}^{\text{T}}\mathbf{W}\mathbf{J}]^{-1}\mathbf{J}^{\text{T}}))} $
    \item error4: error from Monte Carlo simulation.
    \item warning: string with 3 character which indicates:
        \begin{itemize} 
            \item \textbf{str[0]}: number of available measurements, \textbf{1}: for 1 measurement, \textbf{2}: for 2 measurement, \textbf{3}: for 3 or more measurements.
            \item \textbf{str[1]}: \textbf{1}: if one measurement is in the same band and is within the 10 days time windows, \textbf{0} if not.
            \item \textbf{str[2]} \textbf{1}: if there is at least 1 measurement before an after, \textbf{0} if not.
            \item By default, all 3 character are 4.
        \end{itemize}
    \item notms: 0 not enough measurement ($<$3), 1 otherwise.
    \item verbose: string field with the logging of the algorithm. By default \emph{empty}.
\end{itemize}

\section{Client}
In order to simplify the interaction with the web service, a client was created.
This client has two dependencies: astropy and numpy. An example of usage:

\begingroup
\fontsize{8pt}{10pt}\selectfont
\lstset{language=Python}
\begin{lstlisting}
def main():
    #Using default URL
    flux = fluxEstimation('3c279', '04-Apr-2014', '231.435E9,240.435E9', verbose = 'false', test = 'false')

    #Using custom URL
    #bender_url = 'http://bender.csrg.cl:2121/bfs-0.1/ssap'
    #flux = fluxEstimation('3c279', '04-Apr-2014', '231.435E9,240.435E9', url = DEF_URL_localhost,240.435E9)

    flux.performQuery()
    flux.parseResponse()

    #Iterate over all rows
    for row in flux.data:
        print "----------------------------------"
        print row.asDict()
        #print row.asList()
    print "----------------------------------"

if __name__ == "__main__":
    main()
\end{lstlisting}
\endgroup

\end{document}
